\chapter{Úvod}
Každý linuxový server alebo osobný počítač môže slúžiť na niečo iné. Preto je
veľmi náročné vytvoriť linuxovú distribúciu, ktorá by pokrývala požiadavky
každého a bola optimalizovaná pre všetky operácie. Preto je potrebné systém
nastaviť tak, aby presne vyhovoval naším potrebám a získali sme maximálny výkon
pre naše potreby. Kedže sa jedná a množstvo druhov nastavení, vznikol balíček
$tuned$ \cite{tunedHomepage}, ktorý ich zahrňuje.

\chapter{Popis komponenty tuned}
Balíček $tuned$ je primárne napísaný pre linuxovú distribúciu Fedora a Red Hat
Enterprise Linux. Démon $tuned$ neustále beží, skenuje systém a upravuje
nastavenia podľa potreby. Napríklad najväčšia záťaž na disk je štarte systému
alebo pri ukladaní dat na disk (napríklad filmov). Inak je disk skoro nečinný.
$tuned$ dokáže optimalizovať zápis práve v tej dobe, keď je to potreba. Rovnako
je to aj pri sieťových operáciach.

Súčasťou $tuned$ je aj $ktune$, ktorý ladí systém na základe profilov. Každý z
profilov slúži na iné zameranie a napriamo podľa toho upravuje systém, čím
dosahujeme ešte lepšie výsledky.

\section{Profily}
Profily su hlavne zamerané na CPU, disky, sieť a FSB. Samotný balíček obsahuje
niekoľko predvolených profilov a ako základný profil je po spustení $tuned$
profil $balanced$.

Profily si môžeme aj samy vytvárať. Ak si nie sme istý, čo je potrebné upraviť,
môžeme využiť odporúčania z programu $powertop$ \cite{powertopHomepage} a za
pomoci skriptu $powertop2tuned.py$ si nechať profil vytvoriť automaticky na
základe výstupu z $powertop$. 

\chapter{Plán testovania pre Fedora Linux}
Plan testovania podla IEE829

\section{Test Plan Identifier}
\section{References}

%\section{Introduction}
\section{Úvod}
Na testovanie $tuned$ využijeme pomocnú knižnicu $beakerlib$
\cite{beakerlibHomepage} pre jednoduchšie písanie testov a prehľadnejšiu
interpretáciu dosiahnutých výsledkov. Cieľom testov je analýza, či tuned
profily spĺňajú požadované vlastnosti.

%\section{Test Items}
\section{Testovacie položky}
Napísané testy budú overovať správnu funkcionalitu $tuned$ démona a taktiež
$ktune$ profilov v zameraní na CPU, disky a sieťové operácie. Všetky testy budú
pripravené pre linuxovú distribúciu Fedora 17 \cite{fedoraHomepage}.

%\section{Software Risk Issues}
\section{Softvérové riziká}
V prípade zlyhania niektorých testov môže prísť k poškodeniu už pripojených
diskov alebo k rozladeniu sieťových rozhraní. Preto je vhodné spúšťat sadu
testov na virtuálnom stroji. V prípade vydania novej verzie $tuned$ alebo inej
použitej komponenty je tu riziko, že testy nebudú stabilné a môžu sa správať
nepredvídateľne.

%\section{Features to be Tested}
\section{Čo sa bude testovať}
Testy vytvárajú nové blokové zariadenia pomocou utility $tgtadm$. Tieto nové
disky budú formátované na najpoužívanejšie súborové systémy a testované ich
rýchlosti pri rôznych profiloch tuned.

Taktiež sa budú simulovať rôzne sieťové situácie a prenášať dáta cez rozhrania.
Démon $tuned$ by mál vedieť správne zareagovať a zvýšiť priepustnosť siete.

%\section{Features not to be Tested}
\section{Čo sa nebude testovať}
Pretože testy bežia na virtualizovanom hardvéri, nie všetko je možné otestovať.
Napríklad virtuálny procesor nepodporuje Cx stavy \footnote{Cx sú stavy, v
ktorých sa môže vyskytovať procesor, typicky firmy Intel. Tieto stavy sa volajú
Spiacie stavy (ang. Sleep states) \cite{sleepStates}. Spiace stavy procesoru
slúžia na šetrenie energie.}, ktoré ovplyvňuje profil $latency-performance$ a
preto nie je možné spoľahlivo a automatizovane otestovať ich správu.

\section{Approach}
%\section{Item Pass/Fail Criteria}
\section{Kritéria pre splnenie testov}
Počas testovania so zapnutým démonom $tuned$ by všetky I/O operácie diskov mali
byť rýchlejšie a sieť by mala mať lepšiu priepustnosť.

\section{Suspension Criteria and Resumption Requirements}
\section{Test Deliverables}
\section{Remaining Test Tasks}
\section{Environmental Needs}
\section{Staffing and Training Needs}
\section{Responsibilities}
\section{Schedule}
\section{Planning Risks and Contingencies}
\section{Approvals}
\section{Glossary}

\chapter{}

\chapter{Záver}
Zaver

Citacie, aby sa zobrazili: \cite{Testovani_softwaru} \cite{Software_testing} \cite{Software_testing_patton}
