\emph{Virtual-guest} by mal byť najvhodnejší profil pre virtuálny systém -- to
znamená, aj pre naše testovacie prostredie. Na diskoch upravuje
\emph{readahead} hodnotu a nastavuje ju na 4 krát väčšiu. Je ale dobré vedieť,
že taktiež zároveň načítava nastavenia z profilu \emph{throughput-performance}.

Práve pri tomto profile sú najviac viditeľné rozdiely medzi súborovými
systémami. S použítím \emph{ext3} sme dosiahli zrýchlenie niečo cez 1\,\%, ale s
použitím \emph{ext2} je to až cez 10\,\%.

Toto nastavenie taktiež môžeme označiť za veľmi vhodné pre zrýchlenie diskových
operácií a znova je najväčšie zrýchlenie pri použití s \emph{JFS}, ale
najrýchlejší pri použítí kombinácie \emph{btrfs} a \emph{raid1} ako aj v
predchádzajúcom prípade. 
